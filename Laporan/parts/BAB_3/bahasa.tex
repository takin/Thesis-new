\section{Bahasa} % (fold)
\label{sec:section_name}
menurut \citet{chaer}, bahasa merupakan suatu lambang berupa bunyi, bersifat arbiter, digunakan oleh suatu masyarakat tutur untuk bekerja sama, berkomunikasi dan mengidentifikasi diri. \citet{dardjo} juga mengemukakan bahwa bahasa merupakan sistem simbol lisan yang arbiter yang dipakai oleh anggota suatu masyarakat bahasa untuk berkomunikasi dan berinteraksi antar sesamanya dengan berlandaskan pada budaya yang mereka miliki bersama.

Sebagai sebuah sistem, bahasa memiliki aturan, kaidah atau pola-pola tertentu, baik dalam tata bunyi, tata bentuk kata, maupun tata kalimat. Menurut \citet{dardjo}, bahasa disusun oleh tiga komponen, yaitu: sintaksis, fonologi, dan semantik. Komponen sintaksis adalah komponen yang menangani ihwal yang berkaitan dengan kata, frasa dan kalimat. Komponen fonologi adalah kompoenen yang menangani ihwal yang bekaitan dengan bunyi, sedangkan komponen semantik adalah komponen yang membahas ihwal makna.

\subsection{Kata}
Menurut \citet{chaer} melalui \citet{suryawan} bahwa kata merupakan perwujudan bahasa sehingga bahsa tidak akan ada jika kata tidak ada. Setiap kata mengandung konsep makna dan mempunyai peranan dalam pelaksanaan bahasa. \citet{alwi} mengelompokkan kata ke dalam empat kategori sintaksis utama yaitu: verba, nomina, adjektiva dan adverbia. Selain itu, ia juga memperkenalkan kelompok kata lain yang dinamakan kata tugas yang terdiri dari beberapa sub kelompok yang lebih kecil, yaiut: preposisi, konjungtor dan partikel.

\subsubsection{Verba (kata kerja)}
Kata-kata yang termasuk dalam kelompok kata kerja adalah kata-kata yang dapat diikuti loleh frasa \emph{dengan..} \citet{chaer}. Secara umum, verba berfungsi sebagai predikat atau inti predikat di dalam suatu kalimat \citep{alwi}.

Dilihat dari perilaku semantisnya, verba memiliki makna inheren yang terkandung di dalam verba itu sendiri. Verba dapat mengandung makna inheren \emph{perbuatan (aksi), proses} atau \emph{keadaan yang bukan sifat atau kuantitas} \citep{alwi}.

Secara sintaksi, verba dapat dikelompokkan ke dalam verba taktransitif dan verba transitif. Ketransitifan verba ditentukan oleh dua faktor yaitu: adanya nomina yang berfungsi sebagai ojek dalam kalimat aktif dan kemungkinan objek tersebut berfungsi sebagai subjek dalam kalimat pasif \citep{alwi}.
\begin{enumerate}
	\item \emph{Verba taktransitif}\\
	Adalah verba yang tidak diikuti oleh nomina yang berfungsi sebagai subjek dalam kalimat pasif. Verba taktransitif dibagi ke dalam tiga buah subkelompok yaitu: verba taktransitif tak berpelengkap, verba taktransitif berpelengkap wajib dan verba taktransitif berpelengkap manasuka.
	\item \emph{Verba transitif}\\
	Adalah verba yang memerlukan nomina sebagai objek dalam kalimat aktif dan objek tersebut dapat berfungsi sebagai subjek dalam kalimat pasif. Verba transitif dapat dibagi lagi ke dalam tiga kelompok, yaitu: verba ekatransitif, verba dwitransitif dan verba semitransitif.
\end{enumerate}

\subsubsection{Adjektiva (kata sifat)}
Adjektiva adalah kata yang memberikan keterangan yang lebih khusus tentang sesuatu yang  dinyatakan oleh nomina dalam suatu kalimat \citep{alwi}. Apabila ditinjau dari perilaku semantisnya, adjektiva dibagi menjadi dua kelompok yaitu:
\begin{enumerate}
	\item Adjektiva bertaraf, yaitu yang mengungkapkan suatu kualitas. Adjektiva bertaraf dibagi atas (1) adjektiva pemeri sifat, (2) adjektiva ukuran, (3) adjektiva warna, (4) adjektiva waktu, (5) ajdektiva jarak, (6) adjektiva sikap batin dan (7) adjektiva cerapan.
	\item Adjektiva tak bertaraf yang mengungkapkan keanggotaan dalam suatu golongan.
\end{enumerate}
Sedangkan dari segi sintaksisnya, adjektiva dapat berperilaku sebagi fungsi atribut, fungsi predikat, dan fungsi keterangan.

\subsubsection{Adverbia (kata keterangan)}
Adverbia adalah kata-kata yang digunakan untuk memberi penjelasan pada kalimat atau bagian kalimat lain, yang sifatnya tidak menerangkan keadaan atau sifat \citep{chaer}. Adverbia perlu dibedakan dalam tataran frasa dan dalam tataran klausa. Dalam tataran frasa, adverbia adalah kata yang menjelaskan verba, adjektiva dan adverbia lain. Sedangkan dalam tataran klausa, adverbia mewatasi atau menjelaskan fungsi-fungsi sintaksis \citep{alwi}.

\subsubsection{Nomina (kata beda)}
Kata benda adalah kata-kata yang dapat diikuti oleh frasa \emph{yang} ... atau \emph{yang sangat} ... \citep{chaer}. Dari segi bentuk morfologisnya, nomina dapat dibagi menjadi nomina dasar dan nomina turunan. Nomina dasar adalah nomina yang terdiri atas satu morfem sedangkan nomina turunan adalah nomina yang diturunkan melalui proses afiksasi, perulangan dan pemajmukan. Dari sisi semantisnya, kata benda mengacu pada manusia, binatang, benda dan konsep atau pengertian. \citet{alwi} mengungkapkan ciri-ciri nomina secara sintaksis adalah sebagai berikut:
\begin{enumerate}
	\item Dalam kalimat yang predikatnya berupa verba, nomina cenderung menduduki posisi sebagai subjek, objek atau pelengkap.
	
	\item Nomina tidak dapat diingkarkan dengan kata tidak.
	
	\item Nomina umumnya dapat diikuti dengan adjektiva, baik secara langsung maupun dengan diantarai oleh kata yang.
\end{enumerate}

\subsubsection{Pronomina}
Pronomina adalah kata yang digunakan untuk mengacu kepada nomina lain. \citet{alwi} melalui \citet{suryawan} mengungkapkan bahwa bahasa Indonesia mengenal tiga macam nomina yaitu:
\begin{enumerate}
	\item Pronomina persona\\
	Pronomina persona adalah pronomina yang digunakan untuk mengacu pada orang. Pronomina persona dapat mengacu pada diri sendiri (pronomina persona pertama), mengacu pada orang yang diajak bicara (pronomina persona kedua), atau mengacu pada orang yang dibicarakan (pronomina persona ketiga).

	\item Pronomina penunjuk\\
	Bahasa indonesia memiliki tiga macam pronomina penunjuk, yaitu: pronomina penunjuk umum, pronomina penunjuk tempat dan pronomina penunjuk ihwal. Pronomina penunjuk umum dapat mandiri sepenuhnya sebagai nomina. Sebagai nomina, pronomina penunjuk umum dapat berfungsi sebagai subjek atau objek kalimat, dan bahkan dapat pula berfungsi sebagai predikat dalam kalimat yang berpredikat nomina.

	\item Pronomina penanya\\
	Pronomina penanya adalah pronomina yang dipakai sebagai pemarkah pertanyaan. Dari segi maknanya, yang ditanyakan dapat mengenai orang, barang, atau pilihan. Ditinjau dari segi bentuknya, kata penanya didasari oleh dua bentuk yaitu: \emph{apa} dan \emph{mana}. Unsur dasar tersebut dapat dikembangkan menjadi bentuk lain seperti apa, siapa, mengapa, kenapa, kapan, dimana, kemana dan bagaimana.
\end{enumerate}

\subsubsection{Numeralia (kata bilangan)}
Numeralia adalah kata yang digunakan untuk menghitung banyaknya maujud (orang, binatang atau barang) dan konsep. Bahasa Indonesia mengenal dua macam numeralia, yaitu: numeralia pokok (numeralia kardinal) dan numeralia tingkat (numeralia ordinal) \citep{alwi}

\subsubsection{Preposisi (kata depan)}
Preposisi dalam bahasa Indonesia berfungsi sebagai penanda hubungan tempat, peruntukan, sebab, kesertaan atau cara, pelaku, waktu, ihwal (persitiwa) dan milik. Dilihat dari perilaku sintaksisnya, preposisi berada di depan nomina, adjektiva atau adverbia, sedangkan jikal dilihat dari bentuk sintaksisnya, preposisi dapat dibagi menjadi preposisi tunggal dan preposisi majemuk \citep{alwi}

\subsubsection{Konjungtor (kata sambung)}
Konjungtor adalah kata tugas yang menghubungkan dua satuan bahasa yang sederajat: kata dengan kata, frasa dengan frasa, atau klausa dengan kalusa. Dilihat dari perilaku sintaksisnya dalam kalimat, konjungtor dapat dibagi menjadi empat kelompok, yaitu: konjungtor koordinatif, konjungtor subordinatif, konjungtor korelatif dan konjungtor antar kalimat \citep{alwi}

\subsubsection{Interjeksi (kata seru)}
Interjeksi adalah kata tugas yang digunakan untuk mengungkapkan rasa hati pembicara. Interjeksi umumnya ditempatkan di awal kalimat dan pada penulisannya diikuti oleh tanda koma. Secara struktural, interjeksi tidak bertalian dengan unsur kalimat yang lain \citep{alwi}.

\subsubsection{Artikula}
Artikula adalah kata tugas yang membatasi makna nomina. Bahasa Indonesia membagi artikula ke dalam tiga kelompok, yaitu: artikula yang bersifat gelar, artikula yang mengacu ke makna kelompok, dan artikula yang menominalkan \citep{alwi}.

\subsubsection{Partikel Penegas}
Partikel penegas adalah morfem-morfem yang digunakan untuk menegaskan \citep{chaer}. Partikel penegas meliputi kata yang tidak tertakluk pada perubahan bentuk dan hanya berfungsi untuk menampilkan unsur yang diiringinya. Bahasa Indonesia memiliki empat partikel pengas, yaitu: \emph{-lah, -kah, -tah} dan \emph{-pun} \citep{alwi}.

\subsection{Frasa}
Frasa adalah gabungan dua buah kata atau lebih yang merupakan satu kesatuan. Tujuan dari penggabungan dua kata atau lebih menjadi satu kesatuan adalah untuk menampung konsep makna yang lebih khas atau lebih tertentu yang tidak dapat diwujudkan dengan sebuah kata saja \citep{chaer}. Secara sintaksis, frasa dapat dikelompokkan ke dalam frasa verbal, frasa nominal, frasa pronominal, frasa numeralia dan frasa preposisional. 

\subsubsection{Frasa Verbal}
Frasa verbal adalah satuan bahasa bukan klausa yang terbentuk dari dua kata atau lebih dengan verba sebagai intinya \citep{alwi}. Suatu bahasa yang dapat mendampingi verba dalam frasa verbal dapat berupa frasa nominal, klausa pemerlengkap dan atau frasa preposisional \citep{lapoliwa}.

\subsubsection{Frasa Nominal}
Frasa nominal dapat dibentuk dari nomina yang didahului oleh frasa numeral dan atau diikuti oleh frasa nominal, frasa verbal, frasa preposisional, klausa dan atau penentu. Frasa nominal juga dapat dibentuk dari klausa (sebagai pemerlengkapan) yang dapat didahului oleh nomina \citep{lapoliwa}
Menurut \citet{alwi}, frasa nominal dapat dibentuk dari nomina dengan memperluas nomina tersebut ke kiri dan atau ke kanan. Perluasan ke kiri dilakukan dengan meletakkan frasa numeralia di depan nomina. Perluasan ke kanan dapat memiliki bebrapa macam bentuk dengan mengikuti kaidah sebagai berikut:
\begin{enumerate}
	\item Suatu inti dapat diikuti oleh suatu nomina lain atau lebih. Rangkaian tersebut dapat ditutup oleh pronomina persona dan pronomina penunjuk \emph{ini} atau \emph{itu}.
	
	\item Suatu inti dapat diikuti oleh adjektiva, pronomina atau frasa pemilikan, dan kemudian ditutup dengan pronomina penunjut \emph{ini} atau \emph{itu}.
	
	\item Jika suatu nomina diikuti oleh adjektiva dan tidak terdapat pewatas lain yang mengikutinya, kata \emph{yang} dapat disisipkan di antara nomina dan adjektiva. Akan tetapi kata \emph{yang} dan adjektiva harus dipindah ke belakang jika di dalam frasa tersebut terdapat pronomina.

	\item Suatu inti dapat diikuti verba tertentu yang pada hakikatnya dapat dipisahkan oleh kata \emph{yang}, \emph{untuk}, atau unsur tertentu.

	\item Suatu inti dapat diperluas dengan aposisi, yaitu frasa nominal yang mempunyai acuan yang sama dengan nomina yang diterangkannya.

	\item Suatu inti dapat diperluas dengan pewatas belakang, yakni kalusa yang dimulai dengan kata \emph{yang}.

	\item Suatu inti dapat diperluas oleh frasa preposisional. Frasa preposisional yang menjadi pewatas nomina tersebut merupakan bagian dari frasa nominal sehingga tidak dapat dipindahkan ke tempat lain. 
\end{enumerate}

\subsubsection{Frasa Pronominal}
Pronomina dapat diperluas menjadi frasa pronominal. Menurut \citet{alwi}, perluasan pronominal menjadi frasa pronominal dapat dilakukan dengan menambahkan numeralia kolektif, menambahkan kata penunjuk, menambahkan kata \emph{sendiri}, menambahkan klausa yang diawali oleh kata \emph{yang}, atau menambahkan frasa nominal yang memiliki funsi apositif.

\subsubsection{Frasa Numeralia}
Menurut \citet{alwi}, numeralia dapat diperlias menjadi frasa numeralia. Frasa numeralia dapat dibentuk dengan menambahkan kata penggolong di belakang numeralia.

\subsubsection{Frasa Preposisional}
Frasa preposisional dibentuk dari preposisidan frasa nominal. Meskipun frasa preposisional mempunyai makna yang beragam, Struktur frasa preposisional relatif seragam \citep{lapoliwa}.

\subsection{Kalimat}
Kalimat adalah satuan bahasa terkecil, dalam wujud lisan atau tulisan, yang mengungkapkan pikiran yang utuh. Dalam wujud lisan, kalimat diucapkan dengan suara naik turun dan keras lembut, disela jeda, dan diakhiri dengan intonasi akhir yang diikuti dengan kesenyapan. Dalam wujud tulisan berhuruf Latin, kalimat dimulai dengan huruf kapital dan diakhiri dengan tanda titik (.), tanda tanya (?), atau tanda seru (!). Sementara itu, di dalamnya dapat disertakan berbagai tanda baca seperti koma (,), titik dua (:), tanda pisah (-) dan spasi \citep{alwi}.

Menurut \citet{alwi}, kalimat merupakan konstruksi sintaksi terbesar yang terdiri dari dua kata atau lebih. Penggabungan dua kata atau lebih dalam satu kalimat menuntut adanya keserasian diantara unsur-unsur tersebut, baik dari segi makna maupun dari segi bentuk.

Kalimat dapat dikelompokkan berdasarkan jumlah kalusa yang menyusun kalimat. Klausa merupakan satuan sintaksis yang terdiri dari dua kata atau lebih yang mengandung unsur predikasi. Menurut \citet{alwi}, berdasarkan jumlah klausanya, kalimat dapat dibedakan menjadi:
\begin{enumerate}
	\item kalimat tunggal\\
	Kalimat tunggal adalah kalimat yang terdiri atas satu klausa. sehingga konstituen untuk unsur subjek dan predikat hanya ada satu atau merupakan datu kesatuan. Kalimat tunggal dapat mengandung unsur manasuka seperti keterangan tempat, waktu dan alat.

	\citet{alwi} melalui \citet{suryawan} membagi kalimat tunggal berdasarkan kategori predikatnya ke dalam kalimat  berpredikat verbal, kalimat berpredikat adjektival, kalimat berpredikat nominal (termasuk pronominal), kalimat berpredikat numeral dan kalimat berpredikat preposisional.

	\item Kalimat majemuk\\
	kalimat majemuk disusun oleh lebih dari satu klausa. Klausa-klausa yang terdapat di dalam kalimat majemuk bertingkat dapat dihubungkan dengan dua cara, yaitu: koordinasi dan subordinasi.

	Koordinasi adalah menghubungkan dua klausa atau lebih yang memiliki kedudukan konstituen sama. Kalimat majemuk yang dibentuk dengan menggunakan koordinasi disebut dengan \emph{kalimat majemuk setara}.

	Subordinasi adalah menguhubungkan dua klausa atau lebih yang kedudukan konstituennya tidak sama. Hubungan yang dibangun dengan menggunakan subordinasi dapat bersifat melengkapi (komplementatif) atau bersifat mewatasi atau menerangkat (atributif). Klimat majemuk yang dibentuk dengan menggunakan subordinasi disebut dengan \emph{kalimat majemuk bertingkat}.
\end{enumerate}

Kalimat dapat dikelompokkan berdasarkan bentuk atau kategori sintaksis dari kalimat. Berdasarkan bentuk atau kategori sintakasisnya, kalimat dapat dibagi ke dalam empat kelompok, yaitu:
\begin{enumerate}
	\item Kalimat berita (kalimat deklaratif)\\
	Kalimat deklaratif umumnya digunakan untuk menyampaikan pernyataan sehingga isinya merupakan berita bagi pendengar atau pembacanya. Kalimat deklaratif tidak memiliki markah khusus, sehingga kalimat berita dapat berupa bentuk apa saja. Dalam bentuk tulisan, kalimat deklaratif diakhiri dengan tanda titik. Dalam bentuk lisan, kalimat berita diakhiri dengan nada turun.

	\item Kalimat perintah (kalimat imperatif)\\
	\citet{dardjo_et_al} mendefiniskan kalimat imperatif sebagai kalimat yang maknanya meberikan perintah untuk melakukan sesuatu. Dalam bentuk tulisan, kalimat imperatif diakhiri dengan menggunakan tanda seru (!). Dalam bentuk lisan, kalimat imperatif ditandai dengan nada yang semakin tinggi.

	\item Kalimat tanya (kalimat interogatif)\\
	\citet{dardjo_et_al} mendefinisikan kalimat interogatif sebagai kalimat yang isinya menanyakan sesuatu atau seseorang. Kalimat interogatif secara formal ditandai dengan kehadiran kata tanya seperti \emph{apa}, \emph{siapa}, \emph{berapa}, \emph{kapan} dan \emph{bagaimana}, dengan atau tanpa partikel penegas \emph{-kah}. Pada bahasa tulis, kalimat interogatif ditandai dengan tanda tanya (?). Pada bahasa lisan, kalimat tanya ditandai dengan suara naik jika kalimat memiliki kata tanya dan suara turun jika kalimat tidak memiliki kat tanya \citep{alwi}.

	Kalimat interogatif umumnya dibentuk dari kalimat deklaratif. Berikut adalah kaidah pembentukan kalimat tanya dari kalimat deklaratif menurut \citet{alwi}:

	\begin{enumerate}
		\item Kalimat tanya dapat dibentuk dari kalimat deklaratif dengan menambahkan pronomina penanya \emph{apa} pada kalimat tersebut. Partikel \emph{-kah} dapat ditambahkan pada pronomina penanya untuk mempertegas pertanyaan tersebut.

		\item Kalimat tanya dapat dibentuk dengan cara mengubah urutan kata dari kalimat deklaratif. Kaidah yang perlu diperhatikan dalam pembentukan kalimat tanya adalah sebagai berikut:
		\begin{enumerate}
			\item Jika dalam kalimat deklaratif terdapat kata seperti \emph{dapt, bisa, harus, sudah} dan \emph{mau}, maka kata tersebut dipindahkan ke awal kalimat dan ditambahkan partikel \emph{-kah}.

			\item Jika predikat dari kalimat berupa nomina atau adjektiva, urutan subjek dan predikatnya dapat dibalik kemudian partikel \emph{-kah} ditambahkan pada frasa yang telah dipindahkan ke awal kalimat.

			\item Jika predikat kalimat berupa verba taktrasitif, ekatransitif atau semitransitif, maka verba beserta objek atau pelengkapnya dapat dipindahkan ke awal kalimat dan kemudian ditambahkan partikel \emph{-kah}.
		\end{enumerate}

		\item Kalimat tanya dapat dibentuk dengan menempatkan kata \emph{bukan/bukankah, (apa/apakah) belum} atau \emph{tidak}.

		\item Kalimat tanya dapat dibentuk dengan mempertahankan urutan kata seperti dalam kalimat deklaratif, tetapi pengucapannya menggunakan intonasi yang naik.

		\item Kalimat tanya dapat dibentuk dengan menggunakan pronomina interogatif seperti \emph{apa, siapa, mengapa, kenapa, kapan, (Ke)berapa, dimana, kemana, dari mana, bagaimana} dan \emph{bilamana}.
	\end{enumerate}

	\item Kalimat seru (kalimat eksklamatif)\\
	Kalimat seru atau kalimat eksklamatif juga dikenal dengan sebutan kalimat interjeksi. Kalimat seru dapat digunakan untuk menyatakan perasaan kagum atau heran \citep{alwi}. \citet{chaer} menyatakan bahwa kalimat seru juga dapat digunakan untuk menyatakan emosi atau perasaan batin yang biasanya terjadi secara tiba-tiba.
\end{enumerate}

Kalimat dapat dikelompokkan berdasarkan kelengkapan unsur-unsur yang menyusun kalimat. Berdasakan kelengkapan unsurnya, kalimat dapat dikelompokkan ke dalam kalimat lengkap (kalimat major) dan kalimat tak lengkap (kalimat minor). Kalimat lengkap adalah kalimat yang memiliki unsur subjek dan predikat. Kalimat tak lengkap adalah kalimat yang tidak memiliki unsur subjek dan/ atau predikat. Kalimat tak lengkap umumnya muncul di dalam wacana dimana unsur yang tidak muncul tersebut sudah diketahui atau disebutkan sebelumnya \citep{alwi}.
% section section_name (end)
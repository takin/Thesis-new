\section{Bahasa} % (fold)
\label{sec:section_name}
menurut \citet{chaer}, bahasa merupakan suatu lambang berupa bunyi, bersifat arbiter, digunakan oleh suatu masyarakat tutur untuk bekerja sama, berkomunikasi dan mengidentifikasi diri. \citet{dardjo} juga mengemukakan bahwa bahasa merupakan sistem simbol lisan yang arbiter yang dipakai oleh anggota suatu masyarakat bahasa untuk berkomunikasi dan berinteraksi antar sesamanya dengan berlandaskan pada budaya yang mereka miliki bersama.

Sebagai sebuah sistem, bahasa memiliki aturan, kaidah atau pola-pola tertentu, baik dalam tata bunyi, tata bentuk kata, maupun tata kalimat. Menurut \citet{dardjo}, bahasa disusun oleh tiga komponen, yaitu: sintaksis, fonologi, dan semantik. Komponen sintaksis adalah komponen yang menangani ihwal yang berkaitan dengan kata, frasa dan kalimat. Komponen fonologi adalah kompoenen yang menangani ihwal yang bekaitan dengan bunyi, sedangkan komponen semantik adalah komponen yang membahas ihwal makna.

\subsection{Kata}
Menurut \citet{chaer} melalui \citet{suryawan} bahwa kata merupakan perwujudan bahasa sehingga bahsa tidak akan ada jika kata tidak ada. Setiap kata mengandung konsep makna dan mempunyai peranan dalam pelaksanaan bahasa. \citet{alwi} mengelompokkan kata ke dalam empat kategori sintaksis utama yaitu: verba, nomina, adjektiva dan adverbia. Selain itu, ia juga memperkenalkan kelompok kata lain yang dinamakan kata tugas yang terdiri dari beberapa sub kelompok yang lebih kecil, yaiut: preposisi, konjungtor dan partikel.

\subsubsection{Verba (kata kerja)}
Kata-kata yang termasuk dalam kelompok kata kerja adalah kata-kata yang dapat diikuti loleh frasa \emph{dengan..} \citet{chaer}. Secara umum, verba berfungsi sebagai predikat atau inti predikat di dalam suatu kalimat \citep{alwi}.

Dilihat dari perilaku semantisnya, verba memiliki makna inheren yang terkandung di dalam verba itu sendiri. Verba dapat mengandung makna inheren \emph{perbuatan (aksi), proses} atau \emph{keadaan yang bukan sifat atau kuantitas} \citep{alwi}.

Secara sintaksi, verba dapat dikelompokkan ke dalam verba taktransitif dan verba transitif. Ketransitifan verba ditentukan oleh dua faktor yaitu: adanya nomina yang berfungsi sebagai ojek dalam kalimat aktif dan kemungkinan objek tersebut berfungsi sebagai subjek dalam kalimat pasif \citep{alwi}.
\begin{enumerate}
	\item \emph{Varba taktransitif}\\
	Adalah verba yang tidak diikuti oleh nomina yang berfungsi sebagai subjek dalam kalimat pasif. Verba taktransitif dibagi ke dalam tiga buah subkelompok yaitu: verba taktransitif tak berpelengkap, verba taktransitif berpelengkap wajib dan verba taktransitif berpelengkap manasuka.
	\item \emph{Verba transitif}\\
	Adalah verba yang memerlukan nomina sebagai objek dalam kalimat aktif dan objek tersebut dapat berfungsi sebagai subjek dalam kalimat pasif. Verba transitif dapat dibagi lagi ke dalam tiga kelompok, yaitu: verba ekatransitif, verba dwitransitif dan verba semitransitif.
\end{enumerate}

\subsubsection{Adjektiva (kata sifat)}
Adjektiva adalah kata yang memberikan keterangan yang lebih khusus tentang sesuatu yang  dinyatakan oleh nomina dalam suatu kalimat \citep{alwi}. Apabila ditinjau dari perilaku semantisnya, adjektiva dibagi menjadi dua kelompok yaitu:
\begin{enumerate}
	\item Adjektiva bertaraf, yaitu yang mengungkapkan suatu kualitas. Adjektiva bertaraf dibagi atas (1) adjektiva pemeri sifat, (2) adjektiva ukuran, (3) adjektiva warna, (4) adjektiva waktu, (5) ajdektiva jarak, (6) adjektiva sikap batin dan (7) adjektiva cerapan.
	\item Adjektiva tak bertaraf yang mengungkapkan keanggotaan dalam suatu golongan.
\end{enumerate}
Sedangkan dari segi sintaksisnya, adjektiva dapat berperilaku sebagi fungsi atribut, fungsi predikat, dan fungsi keterangan.

\subsubsection{Adverbia (kata keterangan)}
Adverbia adalah kata-kata yang digunakan untuk memberi penjelasan pada kalimat atau bagian kalimat lain, yang sifatnya tidak menerangkan keadaan atau sifat \citep{chaer}. Adverbia perlu dibedakan dalam tataran frasa dan dalam tataran klausa. Dalam tataran frasa, adverbia adalah kata yang menjelaskan verba, adjektiva dan adverbia lain. Sedangkan dalam tataran klausa, adverbia mewatasi atau menjelaskan fungsi-fungsi sintaksis \citep{alwi}.

\subsection{Frasa}
\subsection{Kalimat}
% section section_name (end)
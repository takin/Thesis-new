\section{SPARQL \emph{(Sparql Query Language)}}
SPARQL (diucapkan: ``sparkl'') adalah bahasa \emph{query} RDF \emph{(Resource Description Framework)} dan protokol untuk semantik web \citep{liyang_yu}. Secara harfiah, SPARQL adalah bahasa query yang dapat kita gunakan untuk melakukan query terhadap data dalam bentuk RDF dan SPARQL juga menyediakan protokol yang perlu kita ikuti jika ingin melakukan \emph{query} terhadap \emph{remote} RDF \citep{liyang_yu}. Tim Berners-Lee melalui \citet{ducharme} mengungkapkan ``Mencoba menggunakan semantik web tanpa SPARQL sama seperti menggunakan basis data relasional tanpa SQL''.

Rekomendasi W3C \emph{World Wide Web Consortium} mengenai SPARQL terdiri dari tiga bagian, yaitu:

\begin{enumerate}
	\item \emph{SPARQL query language specification} yang membahas mengani inti dari bahasa query SPARQL.
	\item \emph{SPAQRL Query XML Result Format specification} yang membahas mengenai format standar dari kembalian hasil query.
	\item \emph{SPAQRL Protocol for RDF spesification} yang membahas mengenai protokol standar untuk mengakses RDF di lokasi yang berbeda \emph{(remote)}.
\end{enumerate}

Sparql terdiri dari empat buah bentuk query, yaitu (1) SELECT, (2) ASK, (3) DESCRIBE dan (4) CONSTRUCT. Diantara keempat bentuk query tersebut yang paling banyak digunakan adalah SELECT.

\subsection{\emph{SELECT Query}}
SELECT merupakan bentuk query yang paling sering digunakan. Kebanyakan fiturnya juga digunakan pada bentuk query lainnya (ASK, DESCRIBE dan CONSTRUCT). Bentuk dasar \emph{query SELECT} dapat dilihat pada gambar \ref{fig:bentuk_query_select}.
\begin{figure}[hb]
	\centering
	\begin{lstlisting}[language=SQL, xleftmargin=15pt, numbers=left]
 	BASE <URI>
 	PREFIX pref: <URI>
 	...
 	SELECT <variabel1> <variabel2>
 	FROM <endpoint>
 	WHERE {
 		...
 	}\end{lstlisting} 
	\caption{Bentuk dasar \emph{query SELECT} \citep{liyang_yu}}
	\label{fig:bentuk_query_select}
\end{figure}

Query \emph{SELECT} diawali dengan mendefinisikan sebuah \emph{base} URI kemudian diikuti dengan \emph{PREFIX}. Jumlah \emph{PREFIX} tidak dibatasi sesuai dengan jumlah URI yang akan dilibatkan dalam melakukan \emph{query}. \emph{PREFIX} dapat pula tidak disertakan karena bersifat opsional. Jika tidak menggunakan \emph{PREFIX} maka \emph{URI} harus diuliskan dengan lengkap.

Klausa \emph{SELECT} digunakan untuk melakukan \emph{binding} terhadap data untuk menentukan data apa saja yang akan dikembalikan sebagai hasil \emph{query}. Klausa \emph{FROM} diletakkan setelah klausa \emph{SELECT}. Klausa ini berfungsi untuk menentukan alamat \emph{endpoint} yang akan dikenakan query.

Bentuk \emph{query SELECT} sederhana ditunjukkan dalam gambar \ref{fig:sparql_select_1}. Baris pertama mendeklarasikan \emph{base} URI, dimana \emph{base} URI merupakan alamat URI umum yang akan dijadikan acuan oleh semua URI yang dituliskan secara relatif di dalam query. Pada gambar \ref{fig:sparql_select_1}, URI relatif ditunjukkan dalam baris ke-6 dimana alamat URI lengkap dari <\#danbri> adalah <http://danbri.org/foaf.org\#danbri>. Perhatikan pada baris ke dua dalam gambar \ref{fig:sparql_select_1} dapat dihilangkan karena bersifat opsional dan tidak pernah diacu dalam \emph{statement where}.

\begin{figure}[hb]
	\centering
	\begin{lstlisting}[language=SQL, numbers=left]
	base <http://danbri.org/foaf.rdf>
	PREFIX foaf: <http://xmlns.com/foaf/0.1/>

	select * from <http://danbri.org/foaf.rdf>
	where {
		<#danbri> ?propery ?value .
	}\end{lstlisting}
	\caption{Contoh klausa SELECT dalam query SPARQL \citep{liyang_yu}}
	\label{fig:sparql_select_1}
\end{figure}

Contoh lain penggunaan klausa \emph{SELECT} dalam query SPARQL dapat dilihat dalam gambar \ref{fig:sparql_select_2}. \emph{Query} yang ditunjukkan dalam gambar \ref{fig:sparql_select_2} terdiri dari tiga buah \emph{graph pattren} masing-masing ditunjukkan dalam baris 7, 8 dan 9. Berbeda dengan contoh sebelumnya, pada contoh ke dua ini \emph{PREFIX} tidak dapat dihilangkan karena digunakan dalam klausa \emph{where}. 

\begin{figure}[ht]
	\centering
	\begin{lstlisting}[language=SQL,numbers=left]
	base <http://danbri.org/foaf.rdf>
	PREFIX foaf: <http://xmlns.com/foaf/0.1/>
	PREFIX dc: <http://purl.org/dc/elements/1.1/>

	select * from <http://danbri.org/foaf.rdf>
	where {
		<#danbri> foaf:knows ?friend .
		?friend foaf:depiction ?picture .
		?picture dc:format ?imageFormat .
	}\end{lstlisting}
	\caption{Klausa \emph{SELECT} dengan banyak \emph{graph-pattern} \citep{liyang_yu}}
	\label{fig:sparql_select_2}
\end{figure}

\emph{Query} pada gambar \ref{fig:sparql_select_2} akan menampilkan semua variabel yang terdapat di dalam klausa \emph{where}. Jika hanya ingin menampilkan variabel tertentu maka tanda ``*'' pada baris ke enam diganti dengan nama variabel yang ingin ditampilkan. Contoh query seperti ini dapat dilihat pada gambar \ref{fig:sparql_select_3}.

\begin{figure}[hb]
	\centering
	\begin{lstlisting}[language=SQL,numbers=left]
	base <http://danbri.org/foaf.rdf>
	PREFIX foaf: <http://xmlns.com/foaf/0.1/>
	PREFIX dc: <http://purl.org/dc/elements/1.1/>

	select ?friend,?image from <http://danbri.org/foaf.rdf>
	where {
		<#danbri> foaf:knows ?friend .
		?friend foaf:depiction ?picture .
		?picture dc:format ?imageFormat .
	}\end{lstlisting}
	\caption{Query untuk menampilkan nama dan foto}
	\label{fig:sparql_select_3}
\end{figure}

\subsection{\emph{Query} Terhadap Multi-Graph}
Contoh \emph{query} yang telah kita lihat pada sub bab sebelumnya hanya melibatkan \emph{graph} tunggal saja, namun demikian SPARQL memungkinkan kita untuk melakukan \emph{query} terhadap banyak \emph{graph} sekaligus dengan menggunakan metode \emph{named graph}.

Sebelum melakukan query terhadap multi \emph{graph}, hal yang perlu diperhatikan adalah menyiapkan daftar \emph{named graph} yang akan kita \emph{query}. Definisi \emph{named graph} ditunjukkan dalam gambar \ref{fig:definisi_named_graph}. Baris 2 dan 3 pada potongan query dalam gambar \ref{fig:definisi_named_graph} merupakan alamat URI graph yang akan dikenakan query.

\begin{figure}[ht]
	\centering
	\begin{lstlisting}[language=SQL, numbers=left]
	select *
	from named <uri>
	from named <uri>
	...\end{lstlisting}
	\caption{Konstruksi query terhadap multi \emph{named graph} \citep{liyang_yu}}
	\label{fig:definisi_named_graph}
\end{figure}
\section{Gramatika}
Grammar suatu bahasa dapat dilihat sebagai suatu aturan yang menentukan apakah suatu kumpulan kata dapat diterima sebagai kalimat oleh bahasa tersebut. Sebuah bahasa L dapat dijelaskan sebagai satu kumpulan \emph{string}, dimana \emph{string} dibentuk dari bagian terkecil yang disebut dengan \emph{symbol}. Kelompok tertentu v dari \emph{symbol} biasa dikenal sebagai alfabet atau perbendaharaan kata. Sebuah kalimat yang dapat dikenali dibentuk dengan berdasarkan aturan-aturan tata bahasa yang biasa disebut \emph{grammar}.

Sebuah \emph{grammar} G dapat dibentuk dari empat buah \emph{tuple} yaitu: simbol non-terminal, simbol terminal, simbol awalan dan aturan penulisan atau \emph{rule} sehingga dapat ditulisakan sebagai G=(vn, vt, s, p). Contoh pembentukan kalimat dengan mengacu pada \emph{grammar} G sederhana dapat dilihat dalam gambar \ref{fig:pembentukan_grammar_g}.

\begin{figure}[ht]
	\centering
	\begin{lstlisting}[language=Prolog,xleftmargin=0pt]
		DictJenis = {Kata_Benda,Kata_Kerja,Frasa_Benda,Frasa_Kerja,Keterangan}
		DictKata = {Orang,Makan,Telur,Ayam,Terbang,Tinggi}
		
		dengan aturan:
	
		s           -> Frasa_Benda Frasa_Kerja
		Frasa_Benda -> Kata_Benda Kata_Benda
		Frasa_Kerja -> Kata_Kerja Keterangan
		Kata_Benda  -> {Orang,Telur,Ayam}
		Kata_Kerja  -> {Makan,Terbang}
		Keterangan  -> {Tinggi}
		
		Dari grammar G dapat dibentuk kalimat:
	
		Orang Makan Ayam
		Ayam Terbang Tinggi
		Orang Terbang Tinggi
		Ayam Makan Orang\end{lstlisting}
\caption{Contoh pembentukan kalimat dengan aturan gramatika}
\label{fig:pembentukan_grammar_g}
\end{figure}

Semua kalimat tersebut apabila dicari pembentukannya melalui \emph{grammar} G dapat dikatakan benar, akan tetapi harus diingat bahwa kalimat dengan \emph{grammar} yang benar hanya berarti benar secara struktural bukan berarti selalu benar dalam makna. Seperti kalimat ketiga yang hanya benar apabila berada dalam konteks `'orang memakai alat`' misalnya pesawat terbang. Sedangkan kalimat keempat malah sama sekali tidak mungkin dapat dimengerti maknanya, selain hanya akan menimbulkan tanda tanya bagi yang membaca. Dari \emph{grammar} kita dapat mempelajari bahasa dari segi struktur dan bukan dari segi makna bahasa itu sediri.

\citet{bar_feigenbaum} membagi \emph{grammar} ke dalam empat kelompok berdasarkan bentuk penulisan aturan produksi yaitu:
\begin{enumerate}
	\item \emph{Recursive enumerable grammar}\\
	\emph{Recursive enumerable grammar} atau tipe 0 merupakan tipe yang tidak membatasi jumlah simbol terminal dan non terminal untuk kedua sisi aturan produksinya. Kemampuan ekspresi \emph{(expressive power)} dari \emph{grammar} tipe ini sama dengan mesin turing.
	\item \emph{Contex-sensitive grammar}\\
	\emph{Context-sensitve grammar} atau \emph{grammar} tipe 1 memiliki ketentuan dimana simbol sisi kanan lebih banyak atau sama dengan simbol sisi kiri. \emph{Grammar} ini mampu merepresentasikan bahasa seperti a\textsuperscript{n}b\textsuperscript{n}c\textsuperscript{n}. Contoh \emph{grammar} tipe ini dapat dilihat dalam gambar \ref{fig:contoh_csg} berikut:
	\begin{figure}[ht]
		\centering
		\captionsetup{width=0.85\textwidth}
		\begin{lstlisting}[xleftmargin=30pt]
			S  -> aSBC
			S  -> aBC
			CB -> BC
			aB -> ab
			bB -> bb
			bC -> bc
			cC -> cc\end{lstlisting}
		\caption{Contoh \emph{Context-sensitive grammar} \citep{bar_feigenbaum}}
		\label{fig:contoh_csg}
	\end{figure}

	\item \emph{Contex-free grammar}\\
	Merupakan \emph{grammar} tipe 2 merupakan \emph{grammar} yang memiliki ciri yaitu sisi kiri hanya memiliki satu buah simbol non terminal. Tiap-tiap aturan produksi dapat membuat simbol non terminal dengan konteks yang bebas pada sisi sebelah kanan. \emph{Context-free grammar} populer digunakan untuk \emph{grammar} bahasa alami meskipun konstruksi beberapa bahasa alami tidak bersifat \emph{context-free}. Contoh \emph{grammar} tipe ini dapat dilihat dalam gambar \ref{fig:contoh_cfg} berikut:
	\begin{figure}[hb]
		\centering
		\captionsetup{width=0.85\textwidth}
		\begin{lstlisting}[xleftmargin=30pt]
			<SENTENCE>    -> <NOUN PHRASE> <VERB PHRASE>
			<NOUN PHRASE> -> <DETERMINER> <NOUN>
			<NOUN PHRASE> -> <NOUN>
			<VERB PHRASE> -> <VERB><NOUN PHRASE>
			<DETERMINER>  -> the
			<NOUN> 		  -> boys
			<NOUN>		  -> apples
			<VERB>		  -> eat\end{lstlisting}
		\caption{Contoh penulisan aturan CFG untuk membentuk kalimat \citep{bar_feigenbaum}}
		\label{fig:contoh_cfg}
	\end{figure}


	\item \emph{Regular grammar}\\
	\emph{Regular grammar} atau tipe 3 adalah \emph{grammar} yang memiliki batasan yang paling ketat. Masing-masing aturan produksi memiliki sebuah simbol non terminal pada sisi kiri. Masing-masing aturan produksi memiliki sebuah simbol terminal yang secara opsional dapat diikuti oleh sebuah simbol non terminal pada sisi kanan.
\end{enumerate}
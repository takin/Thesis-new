\section{Ontologi}
Ontologi memiliki peranan penting dalam semantik web. Terdapat berbagai definisi ontologi dalam bidanng semantik web, menurut T.R Gruber melalui \citet*{antoniou}, ontologi adalah \emph{spesifikasi formal dari sebuh konseptualisasi}, sedangkan W3C melalui \citet{liyang_yu} mendefinisikan ontologi sebagai \emph{definisi formal dari sekumpulan term yang digunakan untuk mendeskripsikan dan merepeserentasikan sebuah domain tertentu.}

Ontologi berfungsi sebagai media untuk berbagi pengetahuan dan pemahaman terhadap sesuatu antra domain atau berbagi terminologi yang berbeda namun memliki makna yang sama, misalnya \emph{ZIP Code} sama dengan Kode Wilayah di Indonesia, dengan demikian apabila seseorang mencari dengan menggunakan kata kunci kode wilayah untuk suatu daerah di Amerika misalnya, maka komputer akan dapat memahai bahwa yang dimaksud adalah ZIP Code, demikian juga sebaliknya.

\subsection{Metode pengembangan ontologi}
\citet{fernandez_lopez} melalui \citet*{fonou_huisman} menyebutkan berbagai macam metode yang dapat digunakan untuk pengembangan ontologi, namun demikian \citet{noy_mcguinness} mengungkapkan bahwa tidak ada satu metode yang pasti dalam mengembangkan ontologi. Ia juga mengungkapkan sesungguhnya proses pembuatan ontologi adalah sebuah proses iteratif yang tidak dapat dikerjakan hanya dalam satu tahapan saja, bahkan sangat mungkin pengembangan ontologi terus berlanjut meskipun ontologi sudah digunakan. 

Pemilihan metode pengembangan tergantung pada masing-masing pengembang ontologi, seperti misalnya \citet*{fonou_huisman} memilih menggunakan metode yang dikembangkan oleh \citet*{uschold_king} dengan alasan bahwa metode ini lebih mudah dipahami bagi para pengembang ontologi pemula.

\citet{noy_mcguinness} menawarkan salah satu metode pengembangan ontologi yang didasakan pada pengalaman mereka dalam mengembangkan ontologi. Metode ini paling banyak digunakan dalam pengembangan ontologi. Secara umum tahapan yang harus dilalui dalam pengembangan ontologi adalah sebagai berikut:
\begin{itemize}
	\item Tentukan domain dan ruang lingkup \emph{scope} dari ontologi\\
	Untuk membantu dalam menentukan domain dan ruang lingkup dari ontologi yang akan dibangun, seorang ahli ontologi \emph{ontology engineer} harus dapat menjawab pertanyaan:
	\begin{itemize}
		\item Domain apa yang ingin di-cover oleh ontologi ini ?
		\item Akan digunakan untuk apa ontologi ini ?
		\item Pertanyaan seperti apa yang harus dapat dijawab oleh ontologi ini ?
	\end{itemize}
	Jawaban atas pertanyaan-pertanyaan tersebut mungkin saja dapat berubah selama proses pengembangan ontologi berlangsung, namun setidaknya dapat membantu untuk memastikan ontologi yang akan dibangun tidak keluar dari rung lingkup yang sudah ditetapkan.
	\item Gunakan ontologi yang sudah ada\\
	Sebelum mulai mengembangkan ontologi, ada baiknya untuk mencari apakah ontologi yang akan dibuat sudah pernah dibuat atau belum. Jika sudah ada, apabila memenuhi kriteria yang diinginkan maka sebaiknya menggunakan ontologi tersebut.
	\item Tentukan semua \emph{term} penting dalam ontologi\\
	Tentukan semua \emph{term} baik berupa kelas, objek properti maupun datatype property dari ontologi domain yang akan di-cover.
	\item Buat semua kelas dan strukturnya\\
	Pada tahapan ini, kelas-kelas yang akan di representasikan dalam domain dibuat terlebih dahulu, kemudian diikuti dengan membuat relasi antar kelas-kelas tersebut. Relasi disini termasuk struktur sub dan super kelas. Untuk menentukan struktur relasidapat menggunakan metode \emph{top-down, bottom-up} atau kombinasi keduanya.
	\item Buat properti kelas - \emph{slot}\\
	Setelah proses pembuatan kelas selesai, selanjutnya buat juga properti yang akan digunakan pada kelas-kelas yang sudah dibuat.
	\item Tentukan \emph{facet} dari \emph{slot}\\
	\emph{Facet} menjelaskan mengenai tipe nilai dari kelas, nilai yang diperbolehkan, jumlah yang diperbolehkan \emph{(cardinality)}.
	\item Buat anggota dari kelas \emph{(instance)}\\
	Langkah terakhir adalah membuat anggota atau \emph{instance} dari masing-masing kelas.
\end{itemize}